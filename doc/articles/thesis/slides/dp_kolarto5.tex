\documentclass[t]{beamer}

\usepackage[utf8]{inputenc}
\usepackage{default}
\usepackage[T1]{fontenc}
\usepackage[czech]{babel}

%\usepackage{cutwin}
%\usepackage{wrapfig}
\usepackage{subcaption}
\usepackage{cleveref}

\usetheme{Boadilla}
\usecolortheme{beaver}

\usepackage{hyperref}
\usepackage{graphicx}

%%%Definice

\newcommand{\picFn}[1]{../img/#1}
\newcommand{\picL}{pic}
\newcommand{\rfPic}[1]{(\rf{\picL:#1})}

% 1-File basename
% 2-Label
% 3-Width ratio
\newcommand{\pic}[3]{
\begin{figure}
\centering
\includegraphics[width=#3\linewidth]{\picFn{#1}}
\label{\picL:#2}
\end{figure}
}

%%%%%%%%%%%%%%%%%%%%
%%%%%%%%%%%%%%%%%%%%

%%%Titulni strana

\title[Diplomová práce]{SAT s~diferenciálními rovnicemi}
\subtitle{Diplomová práce}
\author[Kolárik Tomáš (FIT ČVUT)]{
Kolárik Tomáš
\\\textit{kolarto5@fit.cvut.cz}
\\\medskip
Vedoucí práce: doc. Dipl.-Ing. Dr. techn. Stefan Ratschan
}

%%%Uvod

\begin{document}

\begin{frame}
\titlepage
\vfill
\hfill \inserttotalframenumber~slajdů, 10~minut
\end{frame}

%%%%%%%%%%%%%%%%%%%%
%%%%%%%%%%%%%%%%%%%%
%%%%%%%%%%%%%%%%%%%%

\begin{frame}\frametitle{Obsah prezentace}
\begin{enumerate}
\item Motivace.
\item Cíle práce.
\item Výsledky.
\item Ukázka úloh.
\item Realizace.
\end{enumerate}
\end{frame}

%%%%%%%%%%%%%%%%%%%%
%%%%%%%%%%%%%%%%%%%%

\begin{frame}\frametitle{Motivační příklad}
\begin{itemize}
\item Ke~\textit{spolehlivé} činnosti přístroje
   je nutné dodržet provozní teplotu.
\item Řešení: použití \textit{termostatu}.
\item Termostat popíšeme \textit{automatem}.
   \begin{itemize}
   \item Rozsah povolené teploty specifikujeme jako invarianty
      (nejsou na~obrázku).
   \end{itemize}
\item Jak verifikovat správnou funkci termostatu?
   \begin{itemize}
   \item Tj.~dodržení invariant.
   \end{itemize}
\end{itemize}

\vfill

\pic{thermostat_model.pdf}{themostat:model}{0.7}
\end{frame}

%%%%%%%%%%%%%%%%%%%%
%%%%%%%%%%%%%%%%%%%%

\begin{frame}\frametitle{Motivace (1)}
\begin{itemize}
\item Používanou spolehlivou metodou \textit{garance}
   dodržení specifikací je \textbf{formální verifikace}.
   \begin{itemize}
   \item K~tomu je nutné sestavit matematický \textit{model} systému.
   \end{itemize}
\item Dnes hojně využívaný způsob: \textbf{SAT}.
\end{itemize}

\begin{alertblock}{Problém}
\begin{itemize}
\item \textit{Vestavné systémy} interagují
   s~\textit{fyzikálním okolím}.
\item Přirozený popis fyzikálních jevů: \textbf{diferenciální rovnice}.
   \begin{itemize}
   \item Ordinary differential equation (ODE).
   \end{itemize}
\item SAT (ani jeho aritmetická rozšíření) \textit{neumí} ODE.
\end{itemize}
\end{alertblock} %%Problém
\end{frame}

%%%%%%%%%%%%%%%%%%%%
%%%%%%%%%%%%%%%%%%%%

\begin{frame}\frametitle{Motivace (2)}
\begin{block}{Současný stav}
\begin{itemize}
\item Řešice kombinující SAT a soustavy ODE již existují.
\item K~řešení ODE ale používají intervalovou aritmetiku.
\item Důsledky:
   \begin{itemize}
   \item Umožňují intervalové počáteční podmínky ODE.
   \item Garantují maximální dosaženou chybu.
   \item Ale jsou \textbf{pomalé}.
   \end{itemize}
\end{itemize}
\end{block} %%Současný stav

\vfill

\begin{exampleblock}{Cíl práce}
K~řešení ODE použít \textbf{klasické numerické metody}:
\begin{itemize}
\item Vyžadují jednoznačné počáteční podmínky.
\item Garantují (jen) \textit{konvergenci} dosažené chyby.
\item Mohou být méně přesné, ale jsou rychlejší.
\end{itemize}
\end{exampleblock} %%Cíl práce
\end{frame}

%%%%%%%%%%%%%%%%%%%%
%%%%%%%%%%%%%%%%%%%%

\begin{frame}\frametitle{Cíle práce}
\begin{enumerate}
\item Ověřit koncept, který k~řešení ODE používá
   klasické numerické metody.
\item Kombinovat zvolené řešení ODE s~řešením problému SMT.
   \begin{itemize}
   \item SMT: Satisfiability Modulo Theories
      (rozšíření SAT o~aritmetické teorie).
   \end{itemize}
\item Realizovat prototypovou implementaci řešiče SMT a ODE.
\item Srovnat výkonnost prototypu se~stávajícím řešičem dReal.
   \begin{itemize}
   \item dReal pochází z~disertační práce na~Carnegie Mellon University.
   \end{itemize}
\end{enumerate}
\end{frame}

%%%%%%%%%%%%%%%%%%%%
%%%%%%%%%%%%%%%%%%%%

\begin{frame}\frametitle{Výsledky}
\begin{itemize}
\item Náš prototyp si v~některých úlohách počíná mnohem \textbf{rychleji}
   než dReal.
   \begin{itemize}
   \item Zatím se jedná zejména o~úlohy řízené časem.
   \end{itemize}
\item Tj.~podařilo se mi \textbf{potvrdit},
   že zvolený \textbf{koncept} je \textbf{nadějný}
   pro~lepší použití \textbf{v~praxi}.
\item Velký rozdíl ve~výpočtu úloh s~pevnými a intervalovými podmínkami.
   \begin{itemize}
   \item Intervalové podmínky lze aproximovat výčtem hodnot v~logickém součtu.
   \item I~dReal si pak počíná mnohem rychleji.
   \end{itemize}
\end{itemize}
\end{frame}

%%%%%%%%%%%%%%%%%%%%
%%%%%%%%%%%%%%%%%%%%

\begin{frame}\frametitle{Ukázkové úlohy (1)}
\textbf{Termostat}
\begin{itemize}
\item $x$ \dots provozní teplota.
\item Nutné dodržet invariant $70 \leq x \leq 80$.
\item Systém je řízen \textit{časem} o~fixní periodě.
\item Srovnání délky výpočtu dReal a našeho prototypu: $46$ a $0{,}5$ s.
\end{itemize}

\vfill

\begin{figure}
   \begin{subfigure}{.45\textwidth}
      \centering
      \caption{dReal}
      \label{plot:thermostat:dreal}
      \includegraphics[width=1.\linewidth]{\picFn{thermostat_dreal_x.pdf}}
   \end{subfigure}
   \begin{subfigure}{.49\textwidth}
      \centering
      \caption{Náš koncept}
      \label{plot:thermostat:0.4}
      \includegraphics[width=1.\linewidth]{\picFn{thermostat_0_4.pdf}}
   \end{subfigure}
\end{figure}
\end{frame}

%%%%%%%%%%%%%%%%%%%%
%%%%%%%%%%%%%%%%%%%%

\begin{frame}\frametitle{Ukázkové úlohy (2)}
\textbf{Skákající míč}
\begin{itemize}
\item $x$ \dots výška míče, $v$ \dots rychlost.
\item Nutné dodržet invariant $x \geq 0$ nezávisle na~časové periodě.
\item Systém je řízen \textit{událostmi} (změna pád/odraz).
\item Srovnání délky výpočtu dReal a našeho prototypu: $0{,}1$ a $0{,}5$ s.
\item Naše \textit{implementace} je nevhodná
   kvůli nezávislosti na~časové periodě.
\end{itemize}

\vfill

\begin{figure}
   \begin{subfigure}{.49\textwidth}
      \centering
      \caption{dReal}
      \label{plot:ball:dreal}
      \includegraphics[width=1.\linewidth]{\picFn{ball_dreal.pdf}}
   \end{subfigure}
   \begin{subfigure}{.49\textwidth}
      \centering
      \caption{Náš koncept}
      \label{plot:ball:0.2}
      \includegraphics[width=1.\linewidth]{\picFn{ball_0_2.pdf}}
   \end{subfigure}
\end{figure}
\end{frame}

%%%%%%%%%%%%%%%%%%%%
%%%%%%%%%%%%%%%%%%%%

\begin{frame}\frametitle{Realizace}
\begin{itemize}
\item Vstupní jazyk je podobný standardu SMT-LIB.
   \begin{itemize}
   \item Přidány makra pro~parametrizované generování kódu.
   \end{itemize}
\item SMT i~ODE řešič realizovány jako výměnné samostatné komponenty.
\item Centrální komponenta zajišťuje:
   \begin{itemize}
   \item vzájemnou komunikaci řešičů,
   \item prohledání všech potřebných vstupních kombinací.
   \end{itemize}
\end{itemize}

\vfill

\pic{comp_design.pdf}{comp:design}{0.6}
\end{frame}

%%%%%%%%%%%%%%%%%%%%
%%%%%%%%%%%%%%%%%%%%

\begin{frame}\frametitle{Závěr}
\begin{itemize}
\item Cílem bylo aplikovat \textbf{odlišný přístup} v~integraci ODE.
   \begin{itemize}
   \item Použít potenciálně méně přesné, ale rychlejší metody.
   \item Kombinovat ODE s~problémem SAT či jeho rozšířením.
   \end{itemize}
\item Navržený koncept jsem aplikoval v~prototypové implementaci.
\item Prototyp jsem srovnal se~stávajícím řešičem dReal.
   \begin{itemize}
   \item V~některých úlohách jsem dosáhl výrazně \textbf{rychlejšího výpočtu}.
   \item Podařilo se mi \textbf{potvrdit zvolený koncept}.
   \end{itemize}
\item Prototyp přijímá vstupní jazyk, který lze parametrizovat pomocí maker.
\end{itemize}
\end{frame}

%%%%%%%%%%%%%%%%%%%%
%%%%%%%%%%%%%%%%%%%%
%%%%%%%%%%%%%%%%%%%%

% \appendix
% \newcounter{finalframe}
% \setcounter{finalframe}{\value{framenumber}}

% \begin{frame}\frametitle{Další úlohy, které robot vykonává}
% \end{frame}

% \setcounter{framenumber}{\value{finalframe}}

%%%%%%%%%%%%%%%%%%%%
%%%%%%%%%%%%%%%%%%%%
%%%%%%%%%%%%%%%%%%%%

\end{document}
