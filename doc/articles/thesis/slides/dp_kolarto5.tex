\documentclass[t]{beamer}

\beamertemplatenavigationsymbolsempty

\addtobeamertemplate{frametitle}{}{%
   \vspace{-0.3cm}%
}

\usepackage[utf8]{inputenc}
\usepackage{default}
\usepackage[T1]{fontenc}
\usepackage[czech]{babel}

%\usepackage{cutwin}
%\usepackage{wrapfig}
\usepackage{subcaption}
\usepackage{cleveref}

\usetheme{Boadilla}
\usecolortheme{beaver}

\usepackage{hyperref}
\usepackage{graphicx}

%%%Definice

\newcommand{\picFn}[1]{../img/#1}
\newcommand{\picL}{pic}
\newcommand{\rfPic}[1]{(\rf{\picL:#1})}

% 1-File basename
% 2-Label
% 3-Width ratio
\newcommand{\pic}[3]{
\begin{figure}
\centering
\includegraphics[width=#3\linewidth]{\picFn{#1}}
\label{\picL:#2}
\end{figure}
}

%%%%%%%%%%%%%%%%%%%%
%%%%%%%%%%%%%%%%%%%%

%%%Titulni strana

\title[Diplomová práce]{SAT s~diferenciálními rovnicemi}
\subtitle{Diplomová práce}
\author[Kolárik Tomáš (FIT ČVUT)]{
Kolárik Tomáš
\\\textit{kolarto5@fit.cvut.cz}
\\\medskip
Vedoucí práce: doc. Dipl.-Ing. Dr. techn. Stefan Ratschan
}

%%%Uvod

\begin{document}

\begin{frame}
\titlepage
\vfill
\hfill \inserttotalframenumber~slajdů, 10~minut
\end{frame}

%%%%%%%%%%%%%%%%%%%%
%%%%%%%%%%%%%%%%%%%%
%%%%%%%%%%%%%%%%%%%%

\begin{frame}\frametitle{Obsah prezentace}
\begin{enumerate}
%% Komentovat jen prvni bod.
\item Motivace.
\item Cíle práce.
\item Ukázkové úlohy.
\item Model komponent.
\end{enumerate}
\end{frame}

%%%%%%%%%%%%%%%%%%%%
%%%%%%%%%%%%%%%%%%%%

\begin{frame}\frametitle{Motivační příklad}
\begin{itemize}
\item Ke~\textit{spolehlivé} činnosti přístroje
   je nutné dodržet provozní teplotu.
\item Řešení: použití \textit{termostatu}.
\end{itemize}

\pic{thermostat_model.pdf}{themostat:model}{0.7}

\begin{itemize}
\item Termostat popíšeme \textit{automatem}.
   \begin{itemize}
   \item Rozsah povolené teploty: $70 \leq x \leq 80$.
   \end{itemize}
\item Jak \textbf{verifikovat správnou funkci} termostatu?
   \begin{itemize}
   \item Tj.~dodržení rozsahu teploty.
   \end{itemize}
\end{itemize}
\end{frame}

%%%%%%%%%%%%%%%%%%%%
%%%%%%%%%%%%%%%%%%%%

\begin{frame}\frametitle{Motivace}
\begin{itemize}
\item Vestavné systémy typicky vyžadují popis
   pomocí \textbf{diferenciálních~rovnic} (ODE).
   \begin{itemize}
   \item ODE: Ordinary Differential Equation.
   \end{itemize}
\item Samotný problém SAT \textit{neovládá} ODE.
   \begin{itemize}
   \item Ani jeho aritmetická rozšíření je neovládají.
   \end{itemize}
\end{itemize}

\begin{block}{Současný stav}
\begin{itemize}
\item Řešiče kombinující SAT a soustavy ODE již existují.
\item K~řešení ODE ale používají \textit{intervalovou aritmetiku}.
\item Důsledky:
   \begin{itemize}
   \item Umožňují intervalové počáteční podmínky ODE.
   \item Garantují maximální dosaženou chybu.
   \item Ale jsou \textbf{pomalé}.
   \end{itemize}
\end{itemize}
\end{block} %%Současný stav

\vfill

\begin{exampleblock}{Cíl práce}\centering
K~řešení ODE použít \textbf{klasické numerické metody}.
\end{exampleblock} %%Cíl práce
\end{frame}

%%%%%%%%%%%%%%%%%%%%
%%%%%%%%%%%%%%%%%%%%

\begin{frame}\frametitle{Cíle práce}
\begin{enumerate}
\item Ověřit koncept, který k~řešení ODE používá
   \textbf{klasické~numerické~metody}.
   \begin{itemize}
   \item Vyžadují jednoznačné počáteční podmínky.
   \item Mohou být méně přesné, ale jsou \textbf{rychlejší}.
   \end{itemize}
\item Použít zvolené řešení ODE pro~účely \textbf{formální verifikace}.
   \begin{itemize}
      \item Kombinovat ODE a problém SMT.
         \begin{itemize}
         \item SMT: Satisfiability Modulo Theories.
         \item SMT rozšiřuje SAT o~aritmetické teorie.
         \end{itemize}
   \end{itemize}
\item \textbf{Srovnat výkonnost} prototypu se~stávajícím řešičem dReal.
   \begin{itemize}
   \item dReal pochází z~disertační práce na~Carnegie Mellon University.
   %% Vedoucí práce: Edmund M. Clarke, držitel Turingovy ceny
   \end{itemize}
\end{enumerate}
\end{frame}

%%%%%%%%%%%%%%%%%%%%
%%%%%%%%%%%%%%%%%%%%

\begin{frame}\frametitle{Ukázkové úlohy (1)}
\textbf{Termostat}
\begin{itemize}
\item $x$ \dots provozní teplota.
\item Nutné dodržet meze teploty: $70 \leq x \leq 80$.
\item Systém je řízen \textit{časem}:
   \begin{itemize}
   \item předem dané časové okamžiky,
   \item v~nichž dochází k~přechodům a kontrole specifikací.
   \end{itemize}
\item Srovnání délky výpočtu dReal a našeho prototypu:
   \textbf{46} a \textbf{0${,}$5} s.
   \begin{itemize}
   \item Tj. téměř \textbf{stonásobné zrychlení}.
   \end{itemize}
\end{itemize}

\vfill

\begin{figure}
\centering
\label{plot:thermostat}
\includegraphics[width=.55\textwidth]{%
   \picFn{thermostat_0_4_10s.pdf}%
}
\end{figure}
\end{frame}

%%%%%%%%%%%%%%%%%%%%
%%%%%%%%%%%%%%%%%%%%

\begin{frame}\frametitle{Ukázkové úlohy (2)}
\textbf{Skákající míč}
\begin{itemize}
\item $x$ \dots výška míče, $v$ \dots rychlost.
\item Míček musí setrvat nad~podložkou: $x \geq 0$.
\item Systém \textit{není} řízen časem:
   \begin{itemize}
   \item nutno často kontrolovat přechody a specifikace.
   \end{itemize}
\item Srovnání délky výpočtu dReal a našeho prototypu: $0{,}1$ a $0{,}5$ s.
   \begin{itemize}
   \item Naše \textit{implementace} je zatím pro~tyto úlohy neefektivní.
   \end{itemize}
\end{itemize}

\vfill

\begin{figure}
   \begin{subfigure}{.49\textwidth}
      \centering
      \caption{dReal}
      \label{plot:ball:dreal}
      \includegraphics[height=3.3cm, width=\textwidth]{%
         \picFn{ball_dreal.pdf}%
      }
   \end{subfigure}
   \begin{subfigure}{.49\textwidth}
      \centering
      \caption{Náš koncept}
      \label{plot:ball:0.2}
      \includegraphics[height=3.9cm, width=\textwidth]{%
         \picFn{ball_0_2.pdf}%
      }
      \vspace{-0.85cm}
   \end{subfigure}
\end{figure}
\end{frame}

%%%%%%%%%%%%%%%%%%%%
%%%%%%%%%%%%%%%%%%%%

\begin{frame}\frametitle{Pozorování}
\begin{itemize}
\item Náš \textit{prototyp} si v~některých úlohách počíná
   mnohem \textbf{rychleji} než~dReal.
   \begin{itemize}
   \item Zejména v~úlohách řízených časem.
   \end{itemize}
\item Tj.~podařilo se mi \textbf{potvrdit},
   že zvolený \textbf{koncept} je \textbf{nadějný}
   pro~lepší použití \textbf{v~praxi}.
   \begin{itemize}
   \item Průmyslové instance mohou být velmi rozsáhlé.
   \end{itemize}
\end{itemize}
\end{frame}

%%%%%%%%%%%%%%%%%%%%
%%%%%%%%%%%%%%%%%%%%

\begin{frame}\frametitle{Model komponent}
\begin{itemize}
\item Vstupní jazyk je podobný standardu SMT-LIB.
   \begin{itemize}
   \item Přidány makra pro~parametrizované generování kódu.
   \end{itemize}
\item SMT i~ODE řešič realizovány jako výměnné samostatné komponenty.
   \begin{itemize}
   \item Lze použít libovolný SMT řešič konformní s~SMT-LIB standardem.
   \end{itemize}
\end{itemize}

\vfill

\pic{comp_design_api.pdf}{comp:design:api}{0.98}
\end{frame}

%%%%%%%%%%%%%%%%%%%%
%%%%%%%%%%%%%%%%%%%%

\begin{frame}\frametitle{Závěr}
\begin{itemize}
\item Cíl: aplikovat \textbf{odlišný přístup} v~integraci ODE.
   \begin{itemize}
   \item Méně přesné, ale \textbf{rychlejší} metody.
   \end{itemize}
\item Prototyp srovnán s~řešičem dReal.
   \begin{itemize}
   \item Naše řešení je výrazně \textbf{rychlejší} v~některých úlohách.
   \end{itemize}
\item Zvolený \textbf{koncept} se mi podařilo \textbf{potvrdit}.
\item Na~práci budu dále pokračovat.
\end{itemize}
\end{frame}

%%%%%%%%%%%%%%%%%%%%
%%%%%%%%%%%%%%%%%%%%
%%%%%%%%%%%%%%%%%%%%

% \appendix
% \newcounter{finalframe}
% \setcounter{finalframe}{\value{framenumber}}

% \begin{frame}\frametitle{Další úlohy, které robot vykonává}
% \end{frame}

% \setcounter{framenumber}{\value{finalframe}}

%%%%%%%%%%%%%%%%%%%%
%%%%%%%%%%%%%%%%%%%%
%%%%%%%%%%%%%%%%%%%%

\end{document}
